%---------------------------------------------------
% Nombre: Principal.tex  
% 
% Fichero principal del documento. �ste es el punto
% de inicio para la compilaci�n del documento
% 
% Nota: Debe ser compilado 3 veces seguidas para generar
%				el documento correctamente.
%
%---------------------------------------------------

\documentclass[12pt,oneside]{book}

\input ./Configuracion/paquetes.tex   			% inclusi�n de paquetes empleados
\input ./Configuracion/config.tex           % configuraci�n de paquetes
\input ./Configuracion/formato.tex	   % instrucciones de formato generales
\input ./Configuracion/separaciones.tex   	% separaci�n de palabras
\input ./Configuracion/castellano.tex    		% renombrado de titulos autom�ticos a castellano

\usepackage{longtable}
\begin{document}
\newcommand{\tabitem}{~~\llap{\textbullet}~~}
\pagenumbering{alph}											% numeraci�n especial para evitar duplicados
\thispagestyle{empty} 											% la siguiente p�gina no tiene formato
\input ./Prefacio/portada.tex										% inclusi�n de archivo de la portada

\frontmatter													% parte inicial del documento 

\pagenumbering{roman} 											% numeracion romana para las primeras paginas
\thispagestyle{empty} 											% primera pagina sin numero											
\pagenumbering{arabic}											% numeracion arabiga para el resto del documento

\mainmatter																	% parte central del documento 

\tableofcontents														% inclusi�n de tabla de contenidos (�ndice)
%\listoffigures
\listoftables

\input ./Capitulo1/capitulo1.tex     				% inclusi�n de archivo del capitulo 1
\input ./Capitulo2/capitulo2.tex     


\newpage
\mbox{}
\thispagestyle{empty}

\end{document}       
%---------------------------------------------------
